%%% -*- Mode: LaTeX; -*-
%       RCS: $Id: README.tex,v 1.2 2018-09-10 22:55:41 lsipusr Exp $

%%%#+BEGIN: bx:dblock:lcnt:warning-intro :class "memo" :langs "en+fa"

%%%#+END:

%%%#+BEGIN: bx:dblock:lcnt:header-begin :class "memo" :langs "en+fa"
%{{{ DBLOCK-header-begin

\documentclass{article}

\usepackage{hevea} 
%HEVEA\usepackage[utf8]{inputenc}

\htmlhead{
\vspace{0.4in}
}

\htmlfoot{
\bigskip
}


\usepackage{fontspec}
\setmainfont[Mapping=tex-text]{Linux Libertine O}

\usepackage{morefloats}

\usepackage{rcs}
\usepackage{makeidx}
\usepackage{supertabular}
\usepackage{lscape}
\usepackage{array} 
\usepackage{framed}
\usepackage{listings}

\usepackage{color}

\usepackage{hyperref}
\usepackage{url}

\usepackage{fancyhdr}

\usepackage{caption}

\usepackage{fontspec}
\usepackage{xltxtra}
\usepackage{xunicode}
\usepackage{bidi}

\makeatletter
\@ifpackageloaded{caption}{\input{caption-xetex-bidi.def}}{}
\makeatother

\newfontfamily{\persian}[Script=Arabic]{XB Zar}
%\newfontfamily\arabicfont[Script=Arabic,Scale=1]{B Nazanin}%
%\newfontfamily\arabicfontsf[Script=Arabic,Scale=1]{B Nazanin}%
%\newfontinstance{\persian}[Script=Arabic]{B Nazanin}

% for in-line Arabic we need R-L control
\newenvironment{fa}{\beginR\persian}{\endR}

% simple environment for R-L paragraphs
\newenvironment{faPar}
{\everypar={\setbox0\lastbox \beginR
\box0 \persian}}{}

%}}} DBLOCK-header-begin
%%%#+END:

%%%#+BEGIN: bx:dblock:lcnt:style-params :class "memo" :langs "en+fa"
\begin{comment}
*  [[elisp:(org-cycle)][| ]]  *DBLK: style-params*                                       :: [[elisp:(beginning-of-buffer)][Top]] [[elisp:(delete-other-windows)][(1)]]  [[elisp:(org-cycle)][| ]]
\end{comment}
% ===== STYLE PARAMETERS =====

\definecolor{darkred}{rgb}{0.5,0,0}
\definecolor{darkgreen}{rgb}{0,0.5,0}
\definecolor{darkblue}{rgb}{0,0,0.5}

\hypersetup{
    bookmarks=true,         % show bookmarks bar?
    unicode=false,          % non-Latin characters in Acrobat’s bookmarks
    pdftoolbar=true,        % show Acrobat’s toolbar?
    pdfmenubar=true,        % show Acrobat’s menu?
    pdffitwindow=false,     % window fit to page when opened
    pdfstartview={FitH},    % fits the width of the page to the window
    pdftitle={My title},    % title
    pdfauthor={Author},     % author
    pdfsubject={Subject},   % subject of the document
    pdfcreator={Creator},   % creator of the document
    pdfproducer={Producer}, % producer of the document
    pdfkeywords={keyword1} {key2} {key3}, % list of keywords
    pdfnewwindow=true,      % links in new window
    colorlinks=true ,       % false: boxed links; true: colored links
    linkcolor=darkblue,     % color of internal links
    citecolor=red,          % color of links to bibliography
    filecolor=darkgreen,    % color of file links
    urlcolor=darkred        % color of external links
}


\setlength{\textwidth}{6.0in}
\addtolength{\oddsidemargin}{-0.75in}
\addtolength{\evensidemargin}{-0.75in}

\topmargin      0.00 in
\textheight     8.50 in

\setlength{\textwidth}{16.5cm}
\setlength{\topmargin}{-0.3in}
\setlength{\textheight}{8.5in}
\setlength{\oddsidemargin}{0.0cm}
\setlength{\evensidemargin}{0.0cm}


\pagestyle{fancy}
\fancyhead{} % clear all header fields  
%% \fancyhead[C]{{\small  {\tt Work In Progress}}}
\renewcommand{\headrulewidth}{0pt} % no line in header area
\fancyfoot{} % clear all footer fields
%%\fancyfoot[LE,RO]{\thepage}           % page number in "outer" position of footer line
%% \fancyfoot[RE,LO]{{\tt --EARLY DRAFT DOCUMENT--\hspace{20 mm} --Reflects Work In Progress-- }}
\fancyfoot[RE,LO]{}


\parindent 0 true pc

\addtolength{\parskip}{5pt}


%%%#+END:

%%%#+BEGIN: bx:dblock:lcnt:header-end :class "memo" :langs "en+fa"
%{{{ DBLOCK-header-end

\begin{document}
%}}} DBLOCK-header-end

%%%#+END:

%%%#+BEGIN: bx:dblock:lcnt:front-begin :class "memo" :langs "en+fa"
\begin{comment}
*  [[elisp:(org-cycle)][| ]]  *DBLK: front-begin*                                       :: [[elisp:(beginning-of-buffer)][Top]] [[elisp:(delete-other-windows)][(1)]]  [[elisp:(org-cycle)][| ]]
\end{comment}

%%%#+END:

%%%#+BEGIN: bx:dblock:lcnt:copyright :class "memo" :langs "en+fa"
\begin{comment}
*  [[elisp:(org-cycle)][| ]]  *DBLK: copyright*                                       :: [[elisp:(beginning-of-buffer)][Top]] [[elisp:(delete-other-windows)][(1)]]  [[elisp:(org-cycle)][| ]]
\end{comment}

%%%#+END:

%%%#+BEGIN: bx:dblock:lcnt:front-end :class "memo" :langs "en+fa"
\begin{comment}
*  [[elisp:(org-cycle)][| ]]  *DBLK: front-end*                                       :: [[elisp:(beginning-of-buffer)][Top]] [[elisp:(delete-other-windows)][(1)]]  [[elisp:(org-cycle)][| ]]
\end{comment}

%%%#+END:

%%%#+BEGINNOT: bx:dblock:lcnt:main-begin :class "memo" :langs "en+fa"
\begin{comment}
*  [[elisp:(org-cycle)][| ]]  *DBLK: main-begin*                                       :: [[elisp:(beginning-of-buffer)][Top]] [[elisp:(delete-other-windows)][(1)]]  [[elisp:(org-cycle)][| ]]
\end{comment}

\title{unisos.mmwsIcm Library}

%%%#+END:

\thispagestyle{empty}


\bigskip

MM-WS-ICM Library: Machine-to-Machine Web Service Interactive Command Modules (ICM) -- A
set of facilities for developing Performer and Invoker web-services based on Swagger (Open-API) specifications through ICMs.

\section{Sources And Packages}

\subsection{Sources Repositories}

\begin{itemize}
\item GitHub:   \url{https://github.com/bisos-pip/mmwsIcm}
\end{itemize}

\subsection{Packages Repositories}

\begin{itemize}
\item PyPi:   \url{https://pypi.org/project/unisos.mmwsIcm}
\end{itemize}

\section{Support}

For support, criticism, comments and questions; please contact the 
author/maintainer \\
\href{http://mohsen.1.banan.byname.net}{Mohsen Banan} at: \url{http://mohsen.1.banan.byname.net/contact}


\section{Documentation}

Part of ByStar Digital Ecosystem \url{http://www.by-star.net}.

%This module's primary documentation is in  \url{http://www.by-star.net/PLPC/180047}

\begin{itemize}
\item Remote Operations Interactive Command Modules (RO-ICM) -- Best Current (2019) Practices For Web Services Development\\
  \url{http://www.by-star.net/PLPC/180056}
\item A Generalized Swagger (Open-API) Centered Web Services Testing Framework\\
 \url{http://www.by-star.net/PLPC/180057}
\item Interactive Command Modules (ICM) and Players\\
 \url{http://www.by-star.net/PLPC/180050}
\end{itemize}

On the invoker side, a Swagger (Open-API) specification is
digested with bravado and is mapped to command line with ICM.

On the performer side, a Swagger (Open-API) specification is used with
the code-generator to create a consistent starting point.

An ICM can be auto-converted to become a web service.

\section{Binaries And Command-Line Examples}

\begin{itemize}
\item bin/rinvoker.py\\
  A starting point template to be customized for your own swagger file.
\item bin/rinvokerPetstore.py\\
  Provides a list of Petstore example command line invokations.
\item bin/opScnPetstore.py\\
  Points to various scenario files for the Petstore example.
\end{itemize}

\subsection{Remote Invoker (rinvoker-svc.py) Examples}

For the example ``Pet Store Service'' at http://petstore.swagger.io/v2/swagger.json
at command-line (or in bash) you can run: 

\begin{verbatim}
rinvokerPetstore.py
\end{verbatim}

Which will auto generate a complete list of all supported remote opperations 
in the Swagger Service Specification.

You can then invoke any of those remote operations from the command-line, by executing for example:
 
\begin{verbatim}
rinvokerPetstore.py --svcSpec="http://petstore.swagger.io/v2/swagger.json" --resource="pet" --opName="getPetById"  -i rinvoke petId=1
\end{verbatim}

Which will produce something like:

\begin{verbatim}
Operation Status: 200 OK
Operation Result: {   u'category': {   u'id': 0, u'name': u'string'},
    u'id': 1,
    u'name': u'testsw',
    u'photoUrls': [u'string'],
    u'status': u'tttest',
    u'tags': [{   u'id': 0, u'name': u'string'}]}
\end{verbatim}

By turning on verbosity to level 15 (rinvokerPetstore.py -v 15) you can observe 
complete  http traffic as reported by requests library.

\subsection{Operation Scenario (opScn-svc.py) Examples}

For the example ``Pet Store Service'' at http://petstore.swagger.io/v2/swagger.json
using python with RO\_ abstractions you can specify remote invokation and expectations.

To get a list of some example scenatios run:

\begin{verbatim}
opScnPetstore.py
\end{verbatim}

To run a particular example scenario, you can then run:

\begin{verbatim}
opScnPetstore.py  --load /tmp/py2v1/local/lib/python2.7/site-packages/unisos/mmwsIcm-base/opScn-1.py -i roListExpectations
\end{verbatim}

Which will produce something like:

\begin{verbatim}
* ->:: @None@pet@getPetById
** ->:: svcSpec=http://petstore.swagger.io/v2/swagger.json
** ->:: Header Params: None
** ->:: Url Params: 
{   'petId': 1}
** ->:: Body Params: None
* <-:: httpStatus=200 -- httpText=OK -- resultsFormat=json
** <-:: Operation Result: 
{   u'category': {   u'id': 1, u'name': u'dog'},
    u'id': 1,
    u'name': u'Dog1',
    u'photoUrls': [],
    u'status': u'pending',
    u'tags': []}
* ==:: SUCCESS
* XX:: Sleeping For 1 Second
* ->:: @None@pet@getPetById
** ->:: svcSpec=http://petstore.swagger.io/v2/swagger.json
** ->:: Header Params: None
** ->:: Url Params: 
{   'petId': 9999}
** ->:: Body Params: None
* <-:: httpStatus=200 -- httpText=OK -- resultsFormat=json
** <-:: Operation Result: 
{   u'category': {   u'id': 99, u'name': u'SAGScope'},
    u'id': 9999,
    u'name': u'doggie',
    u'photoUrls': [u'string'],
    u'status': u'available',
    u'tags': [{   u'id': 99, u'name': u'SAGTags'}]}
* ==:: SUCCESS

\end{verbatim}


\section{Python Example Usage}

\subsection{Invoker (Client) Development}

\begin{verbatim}
from unisos.mmwsIcm import wsInvokerIcm
from unisos.mmwsIcm import ro
\end{verbatim}

\subsection{Testing Framework}

\begin{verbatim}
from unisos.mmwsIcm import wsInvokerIcm
from unisos.mmwsIcm import ro
\end{verbatim}

\subsection{Performer (Server) Development}

\begin{verbatim}
from unisos.mmwsIcm import wsInvokerIcm
from unisos.mmwsIcm import ro
\end{verbatim}




%%%#+BEGIN: bx:dblock:lcnt:main-end :class "memo" :langs "en+fa"
\begin{comment}
*  [[elisp:(org-cycle)][| ]]  *DBLK: main-end*                                       :: [[elisp:(beginning-of-buffer)][Top]] [[elisp:(delete-other-windows)][(1)]]  [[elisp:(org-cycle)][| ]]
\end{comment}

\end{document}

%%%#+END:

%%%#+BEGIN: bx:dblock:lcnt:latex:end-of-file :class "memo" :langs "en+fa"
%local variables:
%major-mode: latex-mode
%folded-file: nil
%fill-column: 65
%TeX-master: ""
%End:
%%%#+END:
